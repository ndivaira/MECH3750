\documentclass[12pt]{article}

\usepackage{amssymb}
\usepackage{amsmath}
\usepackage{graphicx}
\usepackage{comment}
\usepackage{textcomp}
\usepackage{xcolor}

\usepackage{natbib}

\setlength{\parindent}{0pt}
\usepackage[margin=3cm]{geometry}

%\renewcommand{\myvecf}{\underline{\mathbf{f}}}
\newcommand{\myvec}[1]{\underline{\mathbf{#1}}}

%\begin{document}
%\documentclass{article}
%\usepackage[utf8]{inputenc}
%\usepackage{amsmath}

%\title{FEM Example}
%\author{apr }
%\date{May 2018}

%\usepackage{graphicx}


\setlength{\parindent}{0mm}% Removes the indents.
\begin{document}

\begin{center}
{\Huge   MECH3750 PBL Content Summary}

\vspace{6mm}

{\Huge  Week 7}

\end{center}

\vspace{6mm}

{\Large Content:}
{\begin{itemize}
	\item Numerical Solutions to PDEs
	\begin{itemize}
		\item[--] Explicit Schemes
		\item[--] Implicit Schemes
	\end{itemize}
	\item Parabolic PDEs
\end{itemize}}

\vspace{4mm}

{\Large Upcoming assessment:}
{\begin{itemize}
	\item Assignment 1 (Due Friday)
	\item Problem Sheet 7 (due before Week 8 PBL session)
\end{itemize}}

\vspace{4mm}

{Tutors: Nathan Di Vaira, Alex Muirhead, William Snell, Tristan Samson, Nicholas Maurer, Jakob Ivanhoe, Robert Watt}

%{Tutors: Nathan Di Vaira, Alex Muirhead, William Snell, Tristan Samson, Nicholas Maurer, Jakob Ivanhoe, Robert Watt}

\pagebreak


\section{Numerical Solutions to Parabolic PDEs}

Recall the finite difference approximations we covered earlier in the course. Now we are using these to numerically solve PDES.

\vspace{4mm}

The first type of PDE we are looking at in lectures are parabolic PDEs, the standard form of which is the diffusion equation (written here in 1D),

\vspace{2mm}

$$ \frac{\partial u}{\partial t} = D \frac{\partial^2u}{\partial x^2}. $$
	
\vspace{4mm}	
	
The diffusion equation can be used to model heat conduction by rewriting the diffusion coefficient in terms of the conducting properties of the material, $D=\frac{k}{\rho C_p}$.

\vspace{4mm}

Now, to solve the diffusion equation using finite difference approximations, it is common to use a second order central difference for the spatial derivative,

\vspace{2mm}

$$ \frac{\partial^2u}{\partial x^2} \approx \frac{u_{j+1} - 2u_j + u_{j-1}}{(\Delta x)^2}, $$

\vspace{4mm}
	
where $j$ represents the spatial nodes we have discretised our physical system into.

\subsection{Forward Difference Scheme (Explicit)}

The first way in which we can handle the solution of the time component of the PDE is with a forward difference,

\vspace{2mm}

$$ \frac{\partial u}{\partial t} \approx \frac{u_j^{m+1} - u_j^m}{(\Delta t)}, $$

\vspace{4mm}

where $m$ represents the current time step of the discretised solution.

\vspace{4mm}

Now, substituting the above finite difference approximations for space and time into the original PDE, we obtain an {\it explicit} system of equations to be solved. This means that updated values  of the solution ($m+1$) can be written explicitly in terms of past values ($m$),

$$ u_{j}^{m+1} \approx \sigma u_{j+1}^{m} + (1-2\sigma)u_{j}^{m} + \sigma u_{j-1}^{m} $$

\vspace{4mm}

where $\sigma = D\frac{\Delta t}{(\Delta x)^2}$. These equations can be written concisely in matrix form up to the number of spatial nodes $N$,

\begin{align*}
\begin{bmatrix}
	u_0^{m+1} \\[1ex]
	u_1^{m+1} \\[1ex]
	\vdots \\[1ex]
	\vdots \\[1ex]
	\vdots \\[1ex]
	u_N^{m+1}
\end{bmatrix} 
&=
\begin{bmatrix}
  	1 & 0 & 0 & \dots & \dots & 0 \\[1ex]
    \sigma & 1-2\sigma & \sigma & \ddots & \ddots & \vdots  \\[1ex]
    0 & \sigma & 1-2\sigma & \sigma & \ddots & \vdots \\[1ex]
    \vdots & \ddots & \ddots & \ddots & \ddots & 0 \\[1ex]
    \vdots & \ddots & \ddots & \sigma & 1-2\sigma & \sigma \\[1ex]
    0 & \dots & \dots & 0 & 0 & 1
\end{bmatrix}
\begin{bmatrix}
	u_0^{m} \\[1.2ex]
	u_1^{m} \\[1.2ex]
	\vdots \\[1.2ex]
	\vdots \\[1.2ex]
	\vdots \\[1.2ex]
	u_N^{m}
\end{bmatrix} 
\end{align*}

\vspace{4mm}

This system of equations represents {\it Dirichlet} boundary conditions, where the actual value at the first and last spatial nodes ($j=0$, $j=N$) are specified. We will look at how Neumann boundaries are applied next week.


\subsection{Backward Difference Scheme (Implicit)}

We can also use a backward finite difference approximation for the time derivative,

\vspace{2mm}

$$ \frac{\partial u}{\partial t} \approx \frac{u_j^{m} - u_j^{m-1}}{(\Delta t)}, $$

\vspace{4mm}
	
rendering the update equation,

$$ -\sigma u_{j-1}^{m+1} + (1+2\sigma)u_{j}^{m+1} - \sigma u_{j+1}^{m+1} \approx u_{j}^{m}. $$
	
\vspace{4mm}	
	
We say that this is {\it implicit}, because the solution at the current time step $m$ depends on the solution at the next time step $m+1$. We can again write this system of equations in matrix form,

\begin{align*}
\begin{bmatrix}
  	1 & 0 & 0 & \dots & \dots & 0 \\[1ex]
    -\sigma & 1+2\sigma & -\sigma & \ddots & \ddots & \vdots  \\[1ex]
    0 & -\sigma & 1+2\sigma & -\sigma & \ddots & \vdots \\[1ex]
    \vdots & \ddots & \ddots & \ddots & \ddots & 0 \\[1ex]
    \vdots & \ddots & \ddots & -\sigma & 1+2\sigma & -\sigma \\[1ex]
    0 & \dots & \dots & 0 & 0 & 1
\end{bmatrix}
\begin{bmatrix}
	u_0^{m+1} \\[1.1ex]
	u_1^{m+1} \\[1.1ex]
	\vdots \\[1.1ex]
	\vdots \\[1.1ex]
	\vdots \\[1.1ex]
	u_N^{m+1}
\end{bmatrix}
&=
\begin{bmatrix}
	u_0^{m} \\[1.2ex]
	u_1^{m} \\[1.2ex]
	\vdots \\[1.2ex]
	\vdots \\[1.2ex]
	\vdots \\[1.2ex]
	u_N^{m}
\end{bmatrix} 
\end{align*}

\vspace{4mm}	

this time inverting the matrix to solve for the updated values ($m+1$).


\end{document}