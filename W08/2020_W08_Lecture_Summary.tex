\documentclass[12pt]{article}

\usepackage{amssymb}
\usepackage{amsmath}
\usepackage{graphicx}
\usepackage{comment}
\usepackage{textcomp}
\usepackage{xcolor}

\usepackage{natbib}

\setlength{\parindent}{0pt}
\usepackage[margin=3cm]{geometry}

%\renewcommand{\myvecf}{\underline{\mathbf{f}}}
\newcommand{\myvec}[1]{\underline{\mathbf{#1}}}

%\begin{document}
%\documentclass{article}
%\usepackage[utf8]{inputenc}
%\usepackage{amsmath}

%\title{FEM Example}
%\author{apr }
%\date{May 2018}

%\usepackage{graphicx}


\setlength{\parindent}{0mm}% Removes the indents.
\begin{document}

\begin{center}
{\Huge   MECH3750 PBL Content Summary}

\vspace{6mm}

{\Huge  Week 8}

\end{center}

\vspace{6mm}

{\Large Content:}
{\begin{itemize}
	\item Hyperbolic PDEs
	\begin{itemize}
		\item[--] Wave equation
	\end{itemize}
\end{itemize}}

\vspace{4mm}

{\Large Upcoming assessment:}
{\begin{itemize}
	\item Problem Sheet 8 (due before Week 9 PBL session, after mid-sem break)
\end{itemize}}

\vspace{4mm}

{Tutors: Nathan Di Vaira, Alex Muirhead, William Snell, Tristan Samson, Nicholas Maurer, Jakob Ivanhoe, Robert Watt}

%{Tutors: Nathan Di Vaira, Alex Muirhead, William Snell, Tristan Samson, Nicholas Maurer, Jakob Ivanhoe, Robert Watt}

\pagebreak

\section{Hyperbolic PDEs}

This week in lectures we looked at the 2nd type of PDEs, hyperbolic. Hyperbolic PDEs can be first order (the convection equation),

\vspace{2mm}

$$ \frac{\partial u}{\partial t} + v \frac{\partial u}{\partial x} = 0, $$

\vspace{4mm}

or 2nd order (the wave equation),

\vspace{2mm}

$$ \frac{\partial^2u}{\partial t^2} = c^2 \frac{\partial^2u}{\partial x^2}. $$

\vspace{4mm}

The wave equation is the standard form of hyperbolic PDEs. In PBL {\it Question 1} you should obtain this equation. For oscillations (of a vibrating string), the constant $c^2$ is replaced by $F/\rho A$ (PBL {\it Question 2}).

\subsection{Numerical Solution to Wave Equation}

So how do we solve the wave equation using finite differences? 2nd order central differences are typically used for both the temporal and spatial derivatives,

\vspace{2mm}

$$ \frac{u_{j}^{m-1} - 2u_j^m + u_{j}^{m+1}}{(\Delta t)^2}  = c^2\left(\frac{u_{j-1}^{m} - 2u_j^m + u_{j+1}^{m}}{(\Delta x)^2}\right), $$

\vspace{4mm}

However, it can be seen that this approximation depends on the previous time step ($m-1$), which is a problem at the first time step. To deal with this, we apply a three-point central difference approximation to the first derivative of the initial condition, $U_1(x)$, such that the update equation for the first time step uses $U_1(x)$,

\vspace{2mm}

$$u_j^1 = \frac{\sigma^2}{2}u_{j-1}^0 + (1-\sigma^2)u_j^0 + \frac{\sigma^2}{2}u_{j+1}^0 + \Delta tU_1(x),$$

\vspace{4mm}

recognising that the $u^0$ terms are simply the initial condition for displacement, $U_0(x)$. The general update equation for all subsequent steps is then,

\vspace{2mm}

$$u_j^{m+1} = \sigma^2 u_{j-1}^m + (2-2\sigma^2)u_j^m + \sigma^2 u_{j+1}^m - u_{j}^{m-1},$$

\vspace{4mm}

where $\sigma=c\frac{\Delta t}{\Delta x}$.

\vspace{4mm}

Similar to the diffusion equation last week, these update equations can be written in matrix form (refer to lecture notes).

\subsection{Boundary conditions}

To deal with boundaries in the wave equation, it is easiest to directly apply them at each time step. Common boundary conditions are fixed ends (Dirichlet),

\vspace{2mm}

$$u_0^{m+1} = u_0^m = 0 ; u_N^{m+1} = u_N^m = 0.$$

\vspace{4mm}

or specifying a zero-gradient (Neumann). By taking a 2nd order forward difference at the boundary nodes, the update for the zero-gradient condition can be obtained, 

\vspace{2mm}

$$u_0^{m+1} = \frac{4u_1^{m+1} - u_2^{m+1}}{3} ; u_{N-1}^{m+1} = \frac{4u_{N-2}^{m+1} - u_{N-2}^{N-3}}{3}. $$

\vspace{4mm}

Note, however, that both Dirichlet and Neumann boundaries can be non-zero and/or vary with time.

\end{document}